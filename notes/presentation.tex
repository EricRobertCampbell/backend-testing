\documentclass[t]{beamer}

\usetheme{metropolis}

\usepackage{url}
\usepackage{hyperref}
\hypersetup{
	colorlinks=true,
}
\usepackage{lstautogobble}
\usepackage{listings}
\lstset{numbers=left, frame=l, tabsize=3, basicstyle=\footnotesize, breaklines=true, frame=single}
\lstdefinestyle{python}{language=python}
\lstdefinestyle{output}{numbers=left, frame=l, tabsize=3, basicstyle=\footnotesize, breaklines=true, frame=single}

\newcommand{\code}[1]{\texttt{#1}}

\title{Backend Testing with \code{unittest}, \code{Django}, and \code{Graphene}}
\author{Eric Campbell}
\date{}

\begin{document}
	\begin{frame}
		\maketitle
	\end{frame}

	\section{Unit Testing with \code{unittest}}

	\begin{frame}[fragile]{\code{unittest} Module --- Setup}
		\begin{itemize}
			\item Python comes with a built-in module for testing: \code{unittest}
			\item Several other alternatives, but this one is built in and does what we need
			\item Create a \emph{test suite} by subclassing \code{unittest.TestCase} with a name starting with `Test'
			\item Create a single test as a method of the test class with a name starting with `test\_' --- your test won't be picked up without this
		\end{itemize}
		\begin{lstlisting}[style=python, autogobble]
			import unittest

			class TestSomeStuff(unittest.TestCase):
				test_a_thing(self):
					assertions go here!
				test_another_thing(self):
					some more assertions!
		\end{lstlisting}
	\end{frame}

	\begin{frame}[fragile]{Assertions}
		\begin{itemize}
			\item General format in test: 

				\code{self.assertSomething(params, "Message if the assertion fails")}

			\item Some useful ones:
				\begin{itemize}
					\item \code{self.assertEqual(a, b, "message")} --- whether the two are equal (\code{==})
					\item \code{self.assertTrue(complex condition, "message")} --- whether some complex condition is true
					\item \code{self.assertIn(elem, structure, "message")} --- \code{elem in structure} --- Very useful for testing GraphQL responses
					\item \code{self.assertRaises(Exception, "message")} --- it raises an exception of the given type (use with context provider) 
				\end{itemize}
			\item Also the usual suite of asserting greater than, not equal, is an instance of, \&c.
			\item (Full list at the \href{https://docs.python.org/3/library/unittest.html#unittest.TestCase.assertEqual}{official docs})
		\end{itemize}
	\end{frame}

	\begin{frame}[fragile]{Running}
		\begin{itemize}
			\item To run it:

				\code{python -m unittest test\_some\_stuff.py}
			\item By default, the output is
				\begin{itemize}
					\item Full stop: successful test
					\item E: failed test
				\end{itemize}
			\item Can set the verbosity with \code{--verbose} flag
		\end{itemize}
	\end{frame}

	\begin{frame}[fragile, allowframebreaks]{Example --- Testing an Addition Function}
		Let's test the behaviour of our addition function:
		\lstinputlisting[style=python, title=add.py]{./code/simple/v1-add.py}
		
		It should:
		\begin{enumerate}
			\item Return numbers
			\item Work correctly for \code{int}s
			\item Work correctly for \code{float}s
			\item Fail on anything else
		\end{enumerate}

		\framebreak

		\lstinputlisting[style=python, title=test\_some\_functions.py]{./code/simple/v1-test_some_functions.py}

		\framebreak
	
		\lstinputlisting[style=output, title=Output]{./code/simple/v1-output.txt}
	\end{frame}

	\begin{frame}{Discussion}
		Two things went wrong. One issue was with our function, and one was with the test.
		\begin{enumerate}
			\item Our function doesn't test for type --- it needs to do that!
			\item Our test (floats) didn't account for the fact that floating-point arithmetic is a bit dodgy. We should have used \code{self.assertAlmostEqual}
		\end{enumerate}
	\end{frame}

	\begin{frame}[fragile, allowframebreaks]{Example --- Fixing Issues!}
		Let's fix these issues!

		\framebreak

		\lstinputlisting[style=python, title=add.py]{./code/simple/v2-add.py}

		\lstinputlisting[style=python, title=test\_some\_functions.py]{./code/simple/v2-test_some_functions.py}

		\framebreak

		\lstinputlisting[style=output, title=Output]{./code/simple/v2-output.txt}
	\end{frame}

	\begin{frame}{Your Turn}
		Take a few minutes and write your own test for the related \code{subtract.py}. Think about what sorts of behaviours you want, and try to avoid looking at the examples as much as possible.
	\end{frame}

	\section{Testing \code{Django}}
	
	\begin{frame}{Introduction}
		\begin{itemize}
			\item \code{unittest} module is at the heart of testing Django
			\item For code that tests the actual database, Django provides a subclass of \code{unittest.TestCase} which you subclass for your tests
			\item \code{from django.test import TestCase}
			\item Will automatically start a clean DB instance at the beginning of \emph{each} test and clean up when done
			\item You can freely mix `regular' tests and Django ones in the same test suite
			\item To run these tests, need to use the `test' command:

				\code{python manage.py test the\_file\_you\_are\_testing.py}

			\item You can also omit the filename to test all files it can find or specify a specific app
		\end{itemize}
	\end{frame}

	\begin{frame}[fragile]{\code{setUp} and \code{tearDown}}
		\begin{itemize}
			\item You will often want to have actions performed before and after each test in a suite (e.g. populate / clean the db)
			\item Define a \code{setUp} method: this will be run before each test in the suite
			\item Define a \code{tearDown} method: this will be run after each test in the suite
			\item (These are actually part of the base \code{unittest.TestCase}, so you can also use them in your regular tests)
			\item NB cleanup should be handled in \code{setUp} in case of a failing test
		\end{itemize}
	\end{frame}

	\begin{frame}[fragile]{Example}
			\begin{lstlisting}[style=python, autogobble]
				from Django.test import TestCase
				from app.models import * #need access to your models

				class TestMyModel(TestCase):
					def setUp(self):
						populate the db here with fixtures
					def test_model_works(self):
						make assertions about models here (e.g. test for existence, changes, &c.)
			\end{lstlisting}
	\end{frame}

	\begin{frame}[fragile, shrink]{The App}
		The application we're going to test is a simple listing of books.

		\lstinputlisting[style=python, title=models.py]{./code/django/models.py}
	\end{frame}

	\begin{frame}[fragile, allowframebreaks]{Example --- Testing \code{add\_book\_to\_existing\_author}}
		\lstinputlisting[style=python, firstline=1, lastline=8]{./code/django/utility.py}

		\framebreak

		\lstinputlisting[style=python, title=tests.py]{./code/django/v1-tests.py}

		\framebreak

		\lstinputlisting[style=python, title=tests.py]{./code/django/v1-output.txt}

	\end{frame}

	\begin{frame}[fragile, allowframebreaks]{Example --- Testing the Other One!}
		\code{utility.py} contains another function to add a new book and a new author at the same time. Take a few moments and write some tests for it! Knowing that the signature is \code{add\_book\_with\_author(title, first\_name, last\_name)}, try to write the tests \emph{without} looking at \code{utility.py}

		\framebreak

		\lstinputlisting[style=python, title=tests.py]{./code/django/v2-tests.py}
	\end{frame}


	\section{Testing \code{Graphene}}

	\begin{frame}{Testing Graphene}
		\begin{itemize}
			\item Graphene subclass \code{GraphQLTestCase}
			\item \code{from graphene\_django.utils.testing import GraphQLTestCase}
			\item Allows you to make queries against the schema
			\item In the test:

				\code{self.query('graphql query / mutation', variables=variables)}
			\item Returns a response including the status \code{response.status} and a JSON of the content (\code{response.content})
		\end{itemize}
	\end{frame}

	\begin{frame}{Differences}
		\begin{itemize}
			\item Will need to import your schema
			\item Often useful to set up things common to an entire test suite using \code{setUpClass(cls)} --- a class method
			\item Run once before ALL tests in the suite
		\end{itemize}
	\end{frame}

	\begin{frame}[fragile, allowframebreaks]{Example --- Sample Test}
		\begin{lstlisting}[style = python, autogobble]
			from graphene_django.utils.testing import GraphQLTestCase
			import json

			from somewhere import the models you need
			from somewhere.schema import schema

			class TestSomeQueryOrWhatever(GraphQLTestCase):
				GRAPHQL_SCHEMA = schema

				@classmethod
				def setUpClass(cls):
					super(TestSomeQueryOrWhatever, cls).setUpClass()
					cls.query_string = """
					query {
						stuff {
							etc
						}
					}
					"""
					now I generate the fixtures

				def test_the_query(self):
					query_variables = {'yes': False}
					response = self.query(self.query_string, variables=query_variables)
					decoded = json.loads(response.content)

					assertions go here
		\end{lstlisting}
	\end{frame}

	\begin{frame}{The Setup}
		Here is an explanation of the Graphene side of things:
		\begin{itemize}
			\item Two queries:
				\begin{enumerate}
					\item \code{allBooks}
					\item \code{allAuthors}
				\end{enumerate}
			\item Two mutations, roughly corresponding to the two functions we already looked at
				\begin{enumerate}
					\item \code{createBookWithExistingAuthor}
					\item \code{createBookWithNewAuthor}
				\end{enumerate}
		\end{itemize}
		NB The two functions that we created need to be slighly modified to return the things they create
	\end{frame}

	\begin{frame}[fragile, allowframebreaks]{Graphene Files}
		\lstinputlisting[style=python, title=schema.py]{./code/graphene/schema.py}
		\lstinputlisting[style=python, title=utility.py]{./code/graphene/utility.py}
	\end{frame}

	\begin{frame}[fragile, allowframebreaks]{Example --- Testing Queries}
		Let's test the \code{allBooks} query!	

		\framebreak

		\lstinputlisting[style=python, title=test\_queries\_and\_mutations.py]{./code/graphene/v1-test_queries_and_mutations.py}
	\end{frame}

	\begin{frame}{Your Turn!}
		On your own, try to test the \code{allAuthors} query!
	\end{frame}

	\begin{frame}[fragile, allowframebreaks]{Example --- Testing Mutations}
		Luckily, testing mutations is essentially the same as queries --- the main difference is that you also want to test that the changes to the database have taken as well! This means that we need to test the congruence between:
		\begin{itemize}
			\item The information we sent
			\item The information we received
			\item The information in the database
		\end{itemize}

		We also want to test that it fails in the way that we expect!

		Let's test the \code{createBookWithExistingAuthor} mutation.

		\framebreak

\lstinputlisting[style=python, title=test_queries_and_mutations.py]{./code/graphene/v2-test_queries_and_mutations.py}	
	\end{frame}

	\begin{frame}{Important}
		\begin{itemize}
			\item Test for the status \emph{and} the presence of an \code{error} attribute!
				\begin{itemize}
					\item Status 200: query is syntactically good!
					\item Status 400: query is syntactically invalid!
					\item \code{error} attribute: error once the query has been delivered (i.e. in the resolver)
					\item Careful: a bad response will still have a \code{data} attribute, but now this will contain data about the error
				\end{itemize}
			\item When testing failure modes, be very careful that it is failing for the reason you think! Generally, start with a valid query and go from there
			\item For queries, test database $\iff$ \code{received}
			\item For mutations, test database $\iff$ sent $\iff$ received
		\end{itemize}
	\end{frame}

	\begin{frame}{Your Turn!}
		On your own, try testing the \code{createBookWithNewAuthor} mutation!
	\end{frame}

	\begin{frame}{THE END}
	!	
	\end{frame}
\end{document}
